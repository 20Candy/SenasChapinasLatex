
Tras la culminación del desarrollo de la aplicación ``Señas Chapinas'', se proponen una serie de recomendaciones para continuar optimizando su funcionalidad y aumentar su alcance. Estas recomendaciones están diseñadas para mejorar la experiencia del usuario, expandir las capacidades de la aplicación y asegurar su sostenibilidad a largo plazo.

\begin{enumerate}
    \item \textbf{Ampliar el número de palabras en el diccionario y en el reto diario:} 
    Para incrementar la utilidad de la aplicación, es esencial expandir el número de palabras disponibles en el diccionario, lo cual permitirá a los usuarios acceder a un repertorio más amplio para la traducción y el aprendizaje. Esto también mejoraría la funcionalidad del reto diario, proporcionando una mayor variedad de términos y enriqueciendo la experiencia de los usuarios al interactuar constantemente con nuevas palabras.

    \item \textbf{Contratar un diseñador gráfico para la creación de imágenes coherentes con el diseño de la aplicación:} 
    Es recomendable contratar a un diseñador gráfico que desarrolle las imágenes del diccionario, asegurando que sigan la misma línea estética que el resto de la aplicación. Esto no solo mejorará la funcionalidad del diccionario, sino que también elevará la experiencia visual de los usuarios, presentando imágenes claras, atractivas y alineadas con la identidad visual de la aplicación.

    \item \textbf{Implementar opciones de inicio de sesión con Google, Facebook, entre otros:} 
    La incorporación de opciones de inicio de sesión a través de plataformas populares como Google o Facebook facilitará el acceso de los usuarios, simplificando el proceso de registro y reduciendo las barreras para nuevos usuarios. Esta funcionalidad no solo mejorará la usabilidad de la aplicación, sino que también podría incrementar la tasa de adopción de la aplicación.

    \item \textbf{Incorporar la traducción inversa de español a LENSEGUA:} 
    Para hacer la aplicación más completa, se recomienda implementar la traducción inversa, permitiendo a los usuarios traducir texto de español a LENSEGUA. Esta funcionalidad expandiría significativamente las posibilidades de la aplicación, facilitando que las personas oyentes aprendan y utilicen LENSEGUA de manera más práctica y efectiva.

    \item \textbf{Expandir la aplicación a sistemas iOS:} 
    Para aumentar el alcance de la aplicación, sería recomendable desarrollar una versión para dispositivos iOS. Esta ampliación permitiría a usuarios de la plataforma de Apple beneficiarse de las funcionalidades de la aplicación, logrando un mayor impacto y accesibilidad para personas sordas y oyentes en un rango más amplio de dispositivos.
\end{enumerate}

Estas recomendaciones están diseñadas para fortalecer las funcionalidades actuales de la aplicación, mejorar la experiencia de usuario y abrir nuevas oportunidades de uso y crecimiento en el futuro.
