Desde una edad temprana, fui introducida al mundo de la lengua de señas por mi madre, quien me compartió no solo su experiencia aprendiendo esta lengua, sino también las dificultades y barreras que enfrentan las personas sordas. Recuerdo historias de cómo, en el pasado, a las personas sordas se les obligaba a articular palabras y leer labios, sin permitirles usar su propia forma de comunicación. Estas narrativas sembraron en mí una semilla de curiosidad y compasión que con el tiempo germinaría en un firme deseo de aportar algo significativo a la comunidad sorda.

Este compromiso se vio fortalecido por el invaluable apoyo y los recursos proporcionados por En-Señas Guatemala y ASEDES. A través de su colaboración, obtuve no solo material y datos fundamentales, sino también una profunda inspiración y apoyo constante, elementos esenciales para la realización de este proyecto. Estas instituciones y sus contribuciones han sido vitales para la realización de este proyecto.

Los principales desafíos que enfrenté incluyeron entender realmente las necesidades de las personas sordas. La falta de información disponible en internet me llevó a buscar conocimiento fuera de las fuentes tradicionales, involucrándome directamente con la comunidad sorda a través de clases y numerosas entrevistas. Este acercamiento personal fue crucial para conectar más profundamente con sus experiencias y entender cómo podría ayudar de manera efectiva.

Es mi esperanza que este trabajo ilumine no solo las dificultades diarias que enfrentan las personas sordas, sino también que iniciativas como ``Señas Chapinas'' contribuyan a superar barreras comunicativas. Aspiro a que los usuarios obtengan una visión genuina de la vida en la comunidad sorda y reconozcan la importancia de fomentar un entorno más inclusivo y accesible para todos.

