Quiero expresar mi más sincero agradecimiento a todas las personas que han contribuido a la realización de este proyecto, cada una de las cuales ha sido fundamental en su desarrollo.

Primero, mi gratitud a mi asesor, el Ingeniero Dennis Aldana, por su invaluable guía y apoyo a lo largo de todo el proceso de investigación y redacción de este trabajo. Su dirección experta fue esencial para navegar los retos académicos y prácticos de este proyecto.

Un agradecimiento muy especial a mi madre, quien no solo me introdujo al mundo de la lengua de señas, sino que también me inspiró a embarcarme en este proyecto. Sus historias y experiencias han sido la chispa que encendió mi pasión por hacer una diferencia en la comunidad sorda.

Estoy profundamente agradecida con ASEDES, especialmente a Niurka Waleska Bendfeldt Rosada y Alain de León, por proporcionarme los materiales, las entrevistas y todos los recursos necesarios para llevar a cabo este trabajo. Su colaboración fue indispensable para entender mejor las necesidades y desafíos de la comunidad sorda.

Mi reconocimiento a las alumnas practicantes de ASEDES: Evelyn Cacao, Any Max y Ruth Amézquita, quienes generosamente permitieron que grabáramos sus señas, contribuyendo significativamente a la autenticidad y calidad del contenido de este proyecto.

Finalmente, un agradecimiento especial a Antonio Barrientos, Director General de En-Señas, y a Gabriela Velázquez, maestra de En-Señas. Ambos, además de introducirme prácticamente a la comunidad sorda, me recibieron en En-Señas para que aprendiera la lengua de señas, ayudándome a comprender profundamente sus necesidades y esperanzas. Su apertura y disposición para compartir su conocimiento y experiencia fueron cruciales para este proyecto.

A todos ustedes, mi más profundo respeto y gratitud por su apoyo y contribuciones.